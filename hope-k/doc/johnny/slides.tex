%% SLIDE

\section{Introduzione}

\subsection{Panoramica}

\begin{frame}
 \frametitle{Problemi di classificazione binaria}
 Ogni istanza del dataset � caratterizzata da
 \begin{itemize}
  \item Dato caratteristico del problema: nella trattazione pi� semplice possiamo pensare a vettori di numeri reali, nei casi pi� complessi possono essere tipi di dato strutturato
  \item Etichetta di classificazione binaria, del tipo +1 / -1
 \end{itemize}
 \begin{block}<2->{Obiettivo} 
  Essere in grado di assegnare etichette a nuove istanze del problema
 \end{block}
 \begin{block}<2->{Ostacoli principali}
  \begin{itemize}
   \item In letteratura si distinguono due componenti di errore
    \begin{itemize}
     \item \textbf{Errore di stima} (\textit{overfitting})\\ Troppa fiducia sui dati conosciuti, incapacit� di generalizzare
     \item \textbf{Errore di approssimazione} (\textit{underfitting})\\ Troppa generalit�, incapacit� di discrimare tra istanza e istanza
    \end{itemize}
  \end{itemize}
 \end{block}
\end{frame}

\begin{frame}
 \frametitle{Algoritmi risolutivi}
 Diversi approcci possibili
 \begin{itemize}
  \item Discriminant e Generative Model: basati su considerazioni probabilistiche
  \item Decision Function: a partire da spazio ipotesi e dataset ottenere la migliore funzione discriminante tra esempi positivi e negativi
  \begin{itemize}
   \item Soluzione semplice: percettrone, dati linearmente separabili
   \item Evoluzione: metodi kernel, macchine a vettori di supporto
  \end{itemize}
 \end{itemize}
 \begin{block}<2->{Perch� Kernel Methods ?}
  \begin{itemize}
   \item Nozione di similarit� definibile su qualsiasi tipo di dato
   \item Flessibilit� su dati non linearmente separabili
   \item Indipendenza da algoritmo di apprendimento usato (ammesso sia \textit{kernel-based})
  \end{itemize}
 \end{block}
\end{frame}

\begin{frame}
 \frametitle{Macchine a vettori di supporto - SVM}
 Algoritmo di classificazione, perfettamente integrabile con approccio kernel
 \begin{block}{SVM}
  \begin{itemize}
   \item Generalizzazione iperpiano massimo margine: problema di ottimizzazione quadratico di tipo convesso
   \item Idea \textit{margine soft}: capacit� di generalizzare, risposta migliore in presenza di rumorosit� dei dati
   \item Kernel utilizzabile come plug-in
   \item Disponibili varie implementazioni (es. SVMGist)
  \end{itemize}
 \end{block}	
\end{frame}

\begin{frame}
 \frametitle{Classificazione su NCI Dataset}
 \begin{itemize}
  \item Reso pubblicamente disponibile dal National Cancer Istitute
  \item Capacit� di circa 70000 composti di inibire tumori su 60 linee cellulari umane
  \item Dati riferiti al parametro di concentrazione GI50 (inibizione crescita 50\%)
  \item Per ogni linea cellulare, ogni istanza � caratterizzata da
   \begin{itemize}
    \item Molecola composta da atomi e legami tra atomi: possono essere visti come grafi, etichettati su nodi ed archi
    \item Capacit� di inibizione o meno da parte della molecola (+1, -1)
   \end{itemize}
  \item Dataset ben bilanciato (esempi positivi e negativi circa 50\% l'uno)
 \end{itemize}
 \begin{block}<2->{Scopo}
  Fissata la linea cellulare, essere in grado di prevedere l'effetto pi� o meno inibitorio delle molecole sulla malattia
 \end{block}
\end{frame}

\begin{frame}
 \frametitle{Bilanciamento dataset}
 \begin{center}
  \pgfuseimage{bilanciamento}
 \end{center}
\end{frame}


\section{Kernel fra grafi}

\subsection{Metodologie}

\begin{frame}
 \frametitle{Kernel fra molecole: diverse soluzioni}
 Poich� le istanze del problema sono delle molecole, ai fini della classificazione � necessario definire una misura di
 somiglianza fra di esse. Si noti come senza l'ausilio dei kernel le operazioni tendano di gran lunga a complicarsi.
 
 \begin{block}<2->{Approcci}
  \begin{itemize}
   \item<3-> \alert<3,3>{\textbf{1D kernels:}}\\ Sfruttamento formato SMILES per rappresentazione delle molecole tramite stringhe
   \item<4-> \alert<4,4>{\textbf{2D kernels:}}\\ Studio delle molecole sotto forma di grafi e analisi delle loro caratteristiche
   \begin{itemize}
    \item<4-> Tanimoto Kernel
    \item<4-> MinMax Kernel
   \end{itemize}
   \item<5-> \alert<5,5>{\textbf{3D kernels:}}\\ Basati su rappresentazione 3D delle molecole
  \end{itemize}
 \end{block}
\end{frame}

\subsection{Tanimoto Kernel}

\begin{frame}
 \frametitle{Tanimoto Kernel: descrizione}
 \begin{itemize}
  \item Molecole strutturate in forma di grafi
  \item Ricerca con algoritmo DFS a partire da ciascun nodo, profondit� variabile
  \item Memorizzazione dei cammini (tipi di atomi e di legami)
  \item Conteggio dei percorsi in comune e normalizzazione
 \end{itemize}
 \begin{block}<2->{Perch� normalizzare ?}
  Si discrimina fra molecole con stessi percorsi in comune ma diversa dimensione
 \end{block}
 \begin{block}<3->{Formulazione}
  Data la profondit� \textit{d}, un percorso \textit{path} e le molecole \textbf{x}, \textbf{u}, \textbf{v}, definiamo:
  $$ \phi_{\textit{path}}\left( \textbf{x}\right) = 1\quad sse\;\textit{path} \in \textbf{x} \qquad,\qquad k_d\left( \textbf{u},\textbf{v}\right) = \sum_{\textit{path}}{\phi_{\textit{path}}\left( \textbf{u}\right)\cdot\phi_{\textit{path}}\left( \textbf{v}\right)} $$
  $$ k^{t}_{d}\left( \textbf{u},\textbf{v}\right) = \frac{k_d\left( \textbf{u},\textbf{v}\right)}{k_d\left( \textbf{u},\textbf{u}\right) + k_d\left( \textbf{v},\textbf{v}\right) - k_d\left( \textbf{u},\textbf{v}\right)} $$
 \end{block}
\end{frame}

\begin{frame}
 \frametitle{Tanimoto Kernel: funzionamento}
 \begin{columns}[c]
  \column{0.45\textwidth}
   \textit{\textbf{Molecola 1}}
   \begin{center}
    \pgfuseimage{mol1}
   \end{center}
   \textit{\textbf{Molecola 2}}
   \begin{center}
    \pgfuseimage{mol2}
   \end{center}
  \column{0.50\textwidth}
    \begin{columns}[c]
     \column{0.475\textwidth}
      \begin{block}{\textit{Path} Molecola 1}
       O-C-C-N-C-C\\
       C-N-C-C\\
       C-C-N-C-C\\
       N-C-C\\
       O=C\\
       \ldots
      \end{block}
     \column{0.475\textwidth}
      \begin{block}{\textit{Path} Molecola 2}
       C-C-C-C\\
       O-C-C\\
       O=C-C\\
       O=C\\
       N-C\\
       \ldots
      \end{block}
    \end{columns}
   \vbox{\vskip10pt}
   \begin{block}{Risultati}
    \# \textit{path} mol. 1: \textbf{16}\\ \# \textit{path} mol. 2: \textbf{8}\\ \# \textit{path} comuni: \textbf{6}\\ Valore del Kernel: \textbf{0.3333}
   \end{block}
 \end{columns}
\end{frame}

\subsection{MinMax Kernel}

\begin{frame}
 \frametitle{MinMax Kernel: descrizione}
 \begin{itemize}
  \item Basato sulla struttura del Tanimoto Kernel
  \item Molteplicit� associata a ciascun percorso
  \item Conteggio dei percorsi in comune basato su molteplicit�
  \item Normalizzazione sui dati ottenuti
 \end{itemize}
 \begin{block}<2->{Formulazione}
  Data la profondit� \textit{d}, un percorso \textit{path} e le molecole \textbf{x}, \textbf{u}, \textbf{v}, definiamo:
  $$ \varphi_{\textit{path}}\left( \textbf{x}\right) = \#\; occorrenze\; di\; \textit{path}\; in\; \textbf{x} $$
  $$ k^{m}_{d}\left( \textbf{u},\textbf{v}\right) = \frac{\sum_{\textit{path}}min\left(\varphi_{path}\left( \textbf{u}\right),\varphi_{path}\left( \textbf{v}\right)\right)}{\sum_{\textit{path}}max\left(\varphi_{path}\left( \textbf{u}\right),\varphi_{path}\left( \textbf{v}\right)\right)} $$
 \end{block}
\end{frame}

\begin{frame}
 \frametitle{MinMax Kernel: funzionamento}
 \begin{columns}[c]
  \column{0.45\textwidth}
   \textit{\textbf{Molecola 1}}
   \begin{center}
    \pgfuseimage{mol1}
   \end{center}
   \textit{\textbf{Molecola 2}}
   \begin{center}
    \pgfuseimage{mol2}
   \end{center}
  \column{0.50\textwidth}
    \begin{columns}[c]
     \column{0.475\textwidth}
      \begin{block}{\textit{Path} Molecola 1}
       O-C-C [x2]\\
       C-N-C-C [x1]\\
       C-C [x4]\\
       N-C-C [x1]\\
       O=C [x1]\\
       \ldots
      \end{block}
     \column{0.475\textwidth}
      \begin{block}{\textit{Path} Molecola 2}
       C-C-C-C [x2]\\
       O-C-C [x1]\\
       O=C-C [x1]\\
       O=C [x1]\\
       N-C [x1]\\
       \ldots
      \end{block}
    \end{columns}
   \vbox{\vskip10pt}
   \begin{block}{Risultati}
    \# \textit{path} mol. 1: \textbf{16}\\ \# \textit{path} mol. 2: \textbf{8}\\ \# \textit{path} comuni: \textbf{6}\\ Somma molteplicit� minime: \textbf{9}\\ Somma molteplicit� massime: \textbf{27}\\ Valore del Kernel: \textbf{0.3333}
   \end{block}
 \end{columns}
\end{frame}

\section{Implementazione}

\subsection{Il Codice}

\begin{frame}
 \frametitle{Realizzazione programma (grafi, parser, kernel)}
 Per la generazione del kernel � stato realizzato un software in codice C.\\ \vbox{\vskip10pt} Di seguito le caratteristiche principali.
 \begin{block}<2->{Software Hope-K}
  \begin{itemize}
   \item<3-> Parser ad-hoc per la generazione di grafi a partire da una versione rielaborata (attraverso uno script PERL) del formato mol2
   \item<4-> Capacit� di generazione delle matrici per kernel Tanimoto e MinMax su grafi (sfruttando algoritmi DFS e analisi di cammino)
   \item<5-> Calcolo accuratezza e leave-one-out (loo) a partire dai file ottenuti dall'elaborazione con SVMGist
  \end{itemize}
 \end{block}
\end{frame}

\begin{frame}
 \frametitle{Modalit� di training\&test}
 Esistono varie metodologie per l'analisi ed il confronto dei risultati.\\ Nel caso specifico sono state percorse separatamente due vie differenti, con scopi diversi di seguito illustrati.\\ \vbox{\vskip10pt} 
 \begin{block}<2->{Tecniche: 10 Cross Validation e 80-20 Cross Validation}
  \begin{itemize}
   \item<3-> \alert<3,3>{\textbf{10 Cross Validation:}} la tecnica � stata sfruttata per l'analisi ed il confronto dei due kernel generati, cos� da individuare sia quale dei due presenta un comportamento migliore, quanto quale risulta essere la profondit� ottimale di ricerca
   \item<4->
   \item<4-> \alert<4,4>{\textbf{80-20 Cross Validation:}} basandosi su quanto proposto nell'articolo di riferimento\footnote{\begin{tiny}Kernels for small molecules and the prediction of mutagenicity, toxicity and anti-cancer activity\\ ( Swamidass, Chan, Bruand, Phung, Ralaivola and Baldi )\end{tiny}}, � stata usata questa tecnica per l'estrazione di dati utilizzabili per il confronto con quelli proposti nel documento stesso
  \end{itemize}
 \end{block}
\end{frame}

\begin{frame}
 \frametitle{Script PERL per 80-20 Cross Validation}
 Il kernel ottenuto in forma di matrice quadrata simmetrica non � adatto all'uso diretto per la 80-20 Cross Validation.\\ \vbox{\vskip10pt} A questo scopo � stato realizzato uno script in PERL, utilizzato per la scomposizione della matrice del kernel in matrici di training e test.\\ \vbox{\vskip10pt}
 \begin{block}<2->{Script: \textit{matrix\_maker.pl}}
  \begin{itemize}
   \item<3-> Suddivide la matrice in due sotto-matrici di dimensione variabile
   \item<4-> Effettua ad ogni iterazione lo swapping di colonne scelte in modo casuale
   \item<5-> Genera in maniera randomica una lista di indici utilizzati per la scelta delle colonne che andranno a comporre le due matrici
   \item<6-> Produce una serie di coppie di matrici diverse derivate dall'originale
  \end{itemize}
 \end{block}
\end{frame}

\begin{frame}
 \frametitle{SVMGist}
 Il software Gist contiene strumenti per la classificazione tramite l'uso di macchine a vettori di supporto.\\ Esistono due possibili approcci
 \begin{itemize}
  \item Definire la mappa delle \textit{feature} e fornirne i valori a SVMGist
  \item Definire direttamente la matrice di Gram del kernel (approccio adottato)
 \end{itemize}
 \begin{block}<2->{Fase di \textit{training}}
  Calcolo delle matrici tramite applicativo C e utilizzo di SVMGist al fine di ottenere i pesi ottimali per la funzione di classificazione (risoluzione del problema vincolato di ottimizzazione)
 \end{block}
 \begin{block}<3->{Fase di \textit{test}}
  Utilizzo della funzione di classificazione ottimale per l'esecuzione del test su una porzione del dataset in accordo alla suddivisione adottata (10 Fold Validation, 80/20 Cross Validation)
 \end{block}
\end{frame}

\section{Risultati}

\subsection{Analisi (10 fold validation)}

\begin{frame}
 \frametitle{Tanimoto vs MinMax su dataset 786\_0}
 \begin{center}
  \pgfuseimage{g1}
 \end{center}
\end{frame}

\begin{frame}
 \frametitle{Tanimoto d8 vs MinMax d5 su 10 dataset}
 \begin{center}
  \pgfuseimage{g2}
 \end{center}
\end{frame}

\begin{frame}
 \frametitle{Tanimoto d8 vs MinMax d8 su 10 dataset}
 \begin{center}
  \pgfuseimage{g3}
 \end{center}
\end{frame}

\subsection{Analisi (80/20 fold validation)}

\begin{frame}
 \frametitle{MinMax d10 vs Swamidass MinMax d10 su 6 dataset}
 \begin{center}
  \pgfuseimage{g4}
 \end{center}
\end{frame}

\subsection{Conclusioni}

\begin{frame}
 \frametitle{Risultati finali}
 Di seguito, quanto osservato durante il lavoro
 \begin{itemize}
  \item<2-> Tasso di profondit� basso $\Rightarrow$ scarsa utilit�
  \item<3-> All'aumentare del tasso di profondit�
   \begin{itemize}
    \item MinMax si comporta complessivamente meglio di Tanimoto
    \item Il tempo di computazione cresce inevitabilmente
   \end{itemize}
  \item<4-> Dal confronto con Swamidass � risultato che
   \begin{itemize}
    \item I valori puntuali delle percentuali sono leggermente pi� elevati
    \item Considerando la tolleranza (deviazione standard) sui valori, l'andamento � conforme a quello atteso nella quasi totalit� dei casi
   \end{itemize}
  \item<5-> Nonostante tutto SVMGist non si comporta in modo deterministico ma propone al suo interno un fonte randomica, la quale influisce sulle elaborazioni causando occasionali discostamenti fra i risultati forniti
  \begin{itemize}
   \item I risultati riportati in precedenza sono pertanto soggetti a variazioni
   \item Per i motivi sopra esposti sono state effettuate misurazioni ripetute
  \end{itemize}
 \end{itemize}
\end{frame}
