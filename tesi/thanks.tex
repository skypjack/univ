\thispagestyle{empty}

\footnotesize

Eccomi qua, finalmente a recuperare quel grazie pendente da troppi anni! E allora grazie alla mia vecchietta che m'ha accudito e coccolato e anche a tutti gli altri nonni che non possono vedermi in questo giorno ma che col loro affetto mi hanno sospinto nei miei primi passi incerti in questo mondo e osservato dall'alto mentre diventavo grande.

Un abbraccio ai miei fratelli, che mi sono reso conto valgono proprio un tesoro! Altro che laurea, soldi, lavoro, nulla tiene se mi date un abbraccio proprio quando ne ho bisogno. Vi prendo nel cuore e vi porto con me perché non sarei nulla, altrimenti... Non importa dove sarò un domani, non mi sentirò mai solo grazie a voi... anche se voi tanto lo so che non mi volete bene!! :-D

I miei genitori? Sei lettere son poche per dirvi grazie davvero! Chi c'avrebbe creduto? Io, sei anni fa, ero convinto sarebbe stato un flop e invece, toh! Vedi un po' che Michele c'è oggi. L'avete costruito voi con una manica forse a volte troppo larga e forse a volte troppo stretta ma che non l'ha mai lasciato cadere né soffocato. Oggi non è solo il mio traguardo ma anche di chi mi ha sostenuto e aiutato (e si è pure sorbito tutte le mie giornate storte, da studente mica tanto modello). Ma dove li trovo altri due come voi? Grazie di cuore.

E poi? E poi? La famiglia D'Errico, dai suoceri e la cognata agli zii/e, ai cugini/e e fidanzate/i annesse/i. Insomma quasi un paese in quel del Gargano dove Estate dopo Estate mi hanno accolto con affetto... Solo perché li libero dal peso di una figlia/nipote/cugina! Dovreste pagarmi!! :-D

Giù il cappello davanti al professor Antonio Luchetta che mi ha sopportato non solo per il periodo di tesi ma addirittura dalla triennale! Probabilmente non si aspettava uno come me quando lanciò il sasso ma spero che oggi, in fondo in fondo, sia soddisfatto di questo percorso insieme. Gli devo molti grazie per tutto l'aiuto che mi ha dato e la pazienza e la disponibilità che ha avuto. Fossero tutti così i professori, sarebbe un'altra università!!

Continuando a caso con chi mi viene in mente via via, ci sono gli amici in facoltà. Un grazie a tutti quelli con cui ho condiviso i freddi giorni d'inverno in quel di S. Marta, troppi per elencarvi qua e a tutti quelli che mi hanno accompagnato nel percorso della triennale, che ancora porto nel cuore, soprattutto quegli antipatici pistoiesi di Simone, Matteo e Riccardo. Ovviamente poi, come posso non dire grazie a lui, il mio compagno di viaggio nella specialistica, l'uomo che si è lasciato strapazzare ad ogni progetto, mese dopo mese? Saremo il giorno e la notte, io e te, ma ti devo proprio dire grazie, GJ: e goditelo, perché non capiterà mai più!! :-P

Ma gli amici non finiscono qua: c'è la mia Panzano. La mia odiata e amata Panzano. Ho vissuto momenti fantastici con tutti voi e mi sono sentito abbandonato quando sembrava che nessuno si ricordasse di me, nei primi tempi dell'università. Oggi, però, quando esco e ricevo tante feste e sento il calore di amicizie che non moriranno mai, io, ragazzi, che vi devo dire? Grazie! Non sono più il Michele di 10 anni fa, lo so, ma aver accettato questo mio cambiamento è stato il regalo più grande che mi abbiate mai fatto. Grazie infinite.

Un piccolo grazie (mica tanto piccolo poi) anche a chi ha lavorato con me in questi anni, soprattutto al Tilli che al pari di GJ è stato il mio compagno di lavoro e il mio balocco quando dovevo sfogarmi, senza mai arrabbiarsi ma anzi ridendoci sempre sù. Come farei senza gente così intorno a me, proprio non lo so. Ma loro saranno dello stesso avviso?

\paragraph{}

Mi pare di non aver dimenticato nessuno. Bene, molto bene.

\paragraph{}

Ah, no, forse si, forse qualcuno mi sono dimenticato. Non ho ancora detto grazie a quei due occhi che mi guardano sempre pieni d'amore, a quelle braccia pronte a stringermi in ogni momento, a quella bocca che non mi nega mai una parola amica.\\
Ultima ma non ultima, un grazie a Lilly.\\
Mi hai preso per mano quando sono arrivato confuso in facoltà o forse no, forse ricordo male, mi hai preso per mano quando della mia vita avevo perso il filo e mi ero smarrito e mi hai aiutato a ritrovare la mia strada, ho imparato a camminare di nuovo accanto a te e se oggi spicco il volo e mi libro felice nel cielo è grazie a te che non mi hai mai abbandonato, nella buona e nella cattiva sorte.\\
Siamo seri, ti devo davvero dire grazie? Non servirebbe a rendere l'idea di quello che sei stata e sei tuttora per me.\\
Ma forse un modo c'è per rendere l'idea e quel modo si chiama 5 Marzo 2011. :\#)\\
Ti aspetto... Vedi di non mancare!!

\paragraph{}

E allora grazie a tutti, a chi ho citato e a chi mi sono diligentemente dimenticato. Non ho più tesi per recuperare, quindi a questo giro mi dispiace ma vi dovrete accontentare delle mie più sentite scuse e state certi che in ogni caso il mio GRAZIE è anche per voi, per tutti voi.
