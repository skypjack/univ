\chapter{Appendice A}
In questa appendice viene indicato l'insieme dei simboli riconosciuti da NNSec e la grammatica utilizzata per la descrizione delle reti neurali. Essendo stati utilizzati i programmi JavaCUP e JFlex e avendo questi una sintassi particolare, viene in realt� riportata parte del contenuto dei file sottomessi come ingressi ai programmi sopra citati. Per maggiori informazioni, � consigliato fare riferimento alla documentazione ufficiale dei due programmi sopra citati.

\paragraph{Simboli Terminali}
I simboli basilari, su cui sono costruiti mattone dopo mattone tutti gli altri, sono i seguenti:
\begin{lstlisting}
  nl  =  [\r|\n|\r\n]*
  ws  =  [ \t\f]*
  wtok  =  {nl} | {ws}
  equal  =  {ws}"="{ws}
  sfloat  =  ["+"|"-"]?[0-9]+"."[0-9]*
  integer  =  [0-9]+
  string  =  [a-zA-Z0-9 ]+
  lbracket  =  {wtok}"{"{wtok}
  rbracket  =  {wtok}"}"{wtok}
  endterm	  =  {wtok}";"{wtok}
\end{lstlisting}
A partire da questi, lo spazio di valutazione � diviso in sezioni e alcuni simboli composti possono apparire solo in alcune aree mentre altri assumono significati diversi a seconda dell'area in cui si presentano. L'area principale, dalla quale comincia la valutazione, e i simboli in essa possibili sono di seguito riportati:
\begin{lstlisting}
  <YYINITIAL>
    NET_SEC : "net"{lbracket}
\end{lstlisting}
Quanto sopra impone decisamente che la descrizione relativa ad una rete neurale inizi con il termine \textit{net} seguito da una parentesi graffa aperta, nessun altro termine � accettato. L'unico valore possibile definisce anche la seconda area di valutazione, composta come sotto indicato:
\begin{lstlisting}
  <NET_SEC>
    NAME : "name"{equal}
    DESC : "description"{equal}
    ACTVF : "actv"{equal}
    OUTF : "out"{equal}
    O_NODE_SEC : ["output"{lbracket}|"onode"{lbracket}]
    I_NODE_SEC : ["input"{lbracket}|"inode"{lbracket}]
    H_NODE_SEC : ["hidden"{lbracket}|"hnode"{lbracket}]
    LINK_SEC : "link"{lbracket}
    STRING : {string}
    END_TERM : {endterm}
    NET_SEC_END : {rbracket}
\end{lstlisting}
Le sezioni che seguono direttamente da quella sopra introdotta sono relative ai nodi di ingresso, nodi di uscita e nodi nascosti (ovvero appartenneti ai livelli intermedi) e ai collegamenti presenti fra nodi, rispettivamente.\\
La prima delle sezioni � relativa ai nodi di ingresso:
\begin{lstlisting}
  <I_NODE_SEC>
    NODE_ID : "id"{equal}
    ID_VALUE : {integer}
    END_TERM : {endterm}
    NODE_SEC_END : {rbracket}
\end{lstlisting}
La seconda � invece relativa ai nodi nascosti, o intermedi:
\begin{lstlisting}
  <H_NODE_SEC>
    NODE_ID : "id"{equal}
    ACTVF : "actv"{equal}
    OUTF : "out"{equal}
    GROUP_ID : ["layer"{equal}|"group"{equal}]
    THRESHOLD : "threshold"{equal}
    THRESHOLD_VALUE : {sfloat}
    ID_VALUE : {integer}
    STRING : {string}
    END_TERM : {endterm}
    NODE_SEC_END : {rbracket}
\end{lstlisting}
La terza, ancora, � relativa ai nodi di uscita:
\begin{lstlisting}
  <O_NODE_SEC>
    NODE_ID : "id"{equal}
    ACTVF : "actv"{equal}
    OUTF : "out"{equal}
    THRESHOLD : "threshold"{equal}
    THRESHOLD_VALUE : {sfloat}
    ID_VALUE : {integer}
    STRING : {string}
    END_TERM : {endterm}
    NODE_SEC_END : {rbracket}
\end{lstlisting}
Infine, la sezione riguardante i collegamenti fra nodi:
\begin{lstlisting}
  <LINK_SEC>
    LINK_INPUT_NODE_ID : ["to"{equal}|"hnode"{equal}]
    LINK_OUTPUT_NODE_ID : ["from"{equal}|"tnode"{equal}]
    LINK_WEIGHT : "weight"{equal}
    WEIGHT_VALUE : {sfloat}
    NODE_ID_VALUE : {integer}
    END_TERM : {endterm}
    LINK_SEC_END : {rbracket}
\end{lstlisting}

\paragraph{Grammatica}
La grammatica per la costruzione di reti neurali � di seguito riportata e si basa sui simboli discussi nella sezione precedente. Questa � volutamente semplice ma completa e permettere di descrivere potenzialmente qualsiasi tipo di rete neurale, con le dovute limitazioni in base alle reti ``comprese'' da NNSec. I simboli terminali sono riportati in lettere maiuscole, i simboli non terminali in lettere minuscole.
\begin{lstlisting}
  net_list  ::=  net_list net_expr
                 | net_expr

  net_expr  ::=  NET_SEC net_tokens node link NET_SEC_END

  net_tokens  ::=  net_tokens net_token
                   | net_token

  net_token  ::=  name
                  | desc
                  | actvfunc
                  | outfunc

  name  ::=  NAME STRING END_TERM

  desc  ::=  DESC STRING END_TERM

  actvfunc  ::=  ACTVF STRING END_TERM

  outfunc  ::=  OUTF STRING END_TERM

  node  ::=  node node_expr
             | node_expr

  node_expr  ::=  inode
                  | hnode
                  | onode

  inode  ::=  I_NODE_SEC node_tokens NODE_SEC_END

  hnode  ::=  H_NODE_SEC node_tokens NODE_SEC_END

  onode  ::=  O_NODE_SEC node_tokens NODE_SEC_END

  node_tokens  ::=  node_tokens node_token
                    | node_token

  node_token  ::=  gid
                   | id
                   | nodeactvfunc
                   | nodeoutfunc
                   | threshold

  gid  ::=  GROUP_ID ID_VALUE END_TERM

  id  ::=  NODE_ID ID_VALUE END_TERM

  nodeactvfunc  ::=  ACTVF STRING END_TERM

  nodeoutfunc  ::=  OUTF STRING END_TERM

  threshold  ::=  THRESHOLD THRESHOLD_VALUE END_TERM

  link  ::=  link link_expr
             | link_expr

  link_expr  ::=  LINK_SEC output_ref \
                        input_ref weight LINK_SEC_END

  input_ref  ::=  LINK_INPUT_NODE_ID NODE_ID_VALUE END_TERM

  output_ref  ::=  LINK_OUTPUT_NODE_ID NODE_ID_VALUE END_TERM

  weight  ::=  LINK_WEIGHT WEIGHT_VALUE END_TERM
\end{lstlisting}
\paragraph{}
Grazie alla grammatica utilizzata, descrivere le reti neurale diventa facile e intuitivo per chiunque. Inoltre, per come � stata concepito, NNSec permette di inserire pi� reti neurali in un unico file se si ritiene utile.\\
Insieme al software sono distribuite alcune reti neurali \textit{di prova}, che possono essere utilizzate per testare client e server e dare un'idea di come questi file devono essere effettivamente scritti. Di seguito, un esempio di rete neurale descritta grazie alla grammatica sopra citata:
\begin{lstlisting}
  net {
    name = seno;
    description = ritorna il seno del valore;
    actv = sigmoid;
    inode { id=0; }
    hnode { id = 1; layer = 1; threshold = -2.1814; }
    hnode { id = 2; layer = 1; threshold = -10.9423; }
    hnode { id = 3; layer = 2; threshold = -1.5133; }
    hnode { id = 4; layer = 2; threshold = -2.9522; }
    onode { id = 5; threshold = 0.5348; }
    link { from = 0; to = 1; weight = 3.7558; }
    link { from = 0; to = 2; weight = 12.2071; }
    link { from = 0; to = 3; weight = 14.5007; }
    link { from = 0; to = 4; weight = -2.0168; }
    link { from = 1; to = 5; weight = -13.1986; }
    link { from = 2; to = 5; weight = 6.8320; }
    link { from = 3; to = 5; weight = 5.0785; }
    link { from = 4; to = 5; weight = -1.9914; }
  }
\end{lstlisting}
In tabella \ref{tab:sin} sono riportati i valori associati alla rete neurale riportata. In realt�, questa rete prende come ingressi valori compresi fra 0 e 1 che rappresentano l'angolo tra 0 e 360 gradi, mentre in uscita forniscono il seno dell'angolo nel range $ 0,1 \dots 0,9 $  (il valore da $ 0,1 $ a $ 0,9 $ in uscita deve rappresentare il seno da $ -1 $ a $ +1 $).
\begin{table}[t]
\begin{center}
\begin{tabular}{|c|c|c|c|}
\hline
\textbf{Gradi} & \textbf{Seno} & \textbf{Ingressi} & \textbf{Uscite} \\
\hline
0.0 & 0.0000 & 0.0000 & 0.5000 \\
22.5 & 0.3827 & 0.0625 & 0.6531 \\
45.0 & 0.7071 & 0.1250 & 0.7828 \\
67.5 & 0.9239 & 0.1875 & 0.8696 \\
90.0 & 1.0000 & 0.2500 & 0.9000 \\
112.5 & 0.9239 & 0.3125 & 0.8696 \\
135.0 & 0.7071 & 0.3750 & 0.7828 \\
157.5 & 0.3827 & 0.4375 & 0.6531 \\
180.0 & 0.0000 & 0.5000 & 0.5000 \\
202.5 & -0.3827 & 0.5625 & 0.3469 \\
225.0 & -0.7071 & 0.6250 & 0.2172 \\
247.5 & -0.9239 & 0.6875 & 0.1304 \\
270.0 & -1.0000 & 0.7500 & 0.1000 \\
292.5 & -0.9239 & 0.8125 & 0.1304 \\
315.0 & -0.7071 & 0.8750 & 0.2172 \\
337.5 & -0.3827 & 0.9375 & 0.3469 \\
360.0 & 0.0000 & 1.0000 & 0.5000 \\
\hline
\end{tabular}
\end{center}
\label{tab:sin}
\end{table}
\paragraph{}
Una nota in particolare va fatta sulla flessibilit� implicita, propria di molti linguaggi, che permette di fare cose al di l� di ci� per cui sono stati concepiti grazie a qualche trucco o forzatura delle regole di base. Questo � anche il caso della grammatica introdotta in NNSec. Un esempio molto semplice � la possibilit� di aggiungere implicitamente nodi di bias senza doverli descrivere in modo indipendente, semplicemente ponendo il valore del collegamento associato come soglia del nodo a cui fanno riferimento. Infatti, sia $ t $ la soglia, questa � aggiunta al valore computato per il singolo nodo considerandola come il peso di un collegamento stabilito da un nodo unitario: proprio lo stesso modo in cui viene considerato il valore di bias!\\
Basta lasciar correre la fantasia e anche con la grammatica utilizzata in NNSec, sposata con la struttura del codice, si possono trovare tanti altri trucchi per aggirare problemi o risolvere in modo veloce ed efficiente varie questioni.
