\chapter{Ringraziamenti}

Seduto sul letto di Casa Terronia ho finito di scrivere, di leggere, di rileggere la mia tesi e aspetto solo che i \textit{critici} mi facciano avere le loro revisioni \dots \texttt{Non mi sembra vero!} E visto che la voglia di studiare manca, quale momento migliore per qualche ringraziamento? Da vero informatico andr� in ordine sparso ma consigliando l'uso di un algoritmo per l'ordinamento di questa pagina con complessit� $ O(N\ln N) $.

\paragraph{}
Per�, quanti siete ...

\paragraph{}
Tre anni fa � iniziata l'avventura e i primi che devo ringraziare sono i miei genitori, il mio vecchio e consorte, perch� se non ci credevamo n� io n� voi in fondo ci speravamo tutti un po' che avessi davvero messo la testa a posto. E oggi, guardate qua \dots \textit{Grazie di cuore, grazie di tutto, grazie davvero!} Questo sogno si realizza anche e soprattutto grazie a voi! Sono costoso, forse, ma almeno ne vale la pena (spero)!? Grazie ai miei \texttt{fratelli} e alla mia \texttt{gemellina}, cosa sarei senza di voi: un figlio unico viziato a cui non manca niente, se non due fratelli e una gemellina che non cambierei per niente al mondo. Grazie anche ai \texttt{``parenti acquisiti''}, a tutti quelli che in \textit{Puglia} ogni tanto mi pensano, a tutti quelli che Estate dopo Estate mi accolgono e mi fanno sentire a casa, mi fanno sentire bene. Grazie davvero.

Grazie anche ai miei \textit{amici}, ma non a quelli che si ricordano di me solo quando mi vedono al bar, piuttosto a quelli che si ricordano di me proprio quando non mi vedono al bar!!

Grazie a tutti i \textit{compagni di viaggio} di quest'avventura che � l'universit�, ai \textbf{vecchi} e ai \textbf{nuovi}. Grazie a Droscy e a un amico che spero non perder� mai (chiss� quando mi sarei laureato, senza di te), a Fox e ai suoi problemi idilliaci (quante ore, quante litigate, quante risate), al Booga a cui non ho mai detto grazie per avermi strappato una risata anche all'ospedale mentre piangevo il mio naso rotto, a Trevi\~{n}o (e la sua \~{n} che ci vuole proprio una laurea in ingegneria per digerirla) e a Paolo, il Paolo che ci ha abbandonati, il Paolo del Romito, il Paolo di Geometria e Algebra Lineare, il Paolo che \underline{non � meteora}, il Paolo con cui ho condiviso un primo, splendido anno. Grazie poi a tutti quelli che hanno cominciato con me e incrociato il mio cammino anche solo per un attimo. Ma si, ma si, non mi sono dimenticato, grazie anche alla bionda e al rame: vi ho conosciuti tardi ma meglio tardi che mai, no? E grazie a GJ, un amico a cui avrei potuto chiedere tanto (viste le doti) ma a cui non ho chiesto niente, perch� amico e non miniera d'oro. Grazie agli ultimi arrivati, poi, soprattutto al signor \texttt{rappresentante degli studenti} con cui ho condiviso l'esperienza tesi: ma quante risate ci siamo fatti mentre fuori pioveva? Ultimo, ma non ultimo, chi mi ha prestato i suoi appunti, dato i suoi consigli, accettato come amico non solo nel momento del bisogno quando ho teso la mano ma anche dopo, quando poteva andarsene: � bello sapere che c'� ancora gente cos�, gente che arriva sempre \textit{Just-in Time}!

Ma l'universit� non sono solo \texttt{studenti}, anche \texttt{professori}! Ovviamente grazie ad Alessandro\footnote{\#define Alessandro professore} (aka professor Piva) per avermi accolto nella sua ``famiglia'' e dato la libert� di cui avevo bisogno, e grazie anche a tutto il \texttt{laboratorio comunicazioni e immagini}. Grazie a Claudio perch� mi ha risposto quando neanche sapeva chi fossi e ha continuato a farlo anche dopo averlo scoperto! Grazie pure al professor Luchetta, perch� non mi ci ha mandato nel mio periodo di tesi si, tesi no (ma soprattutto, per il tesi no).

Dulcis in fundo, un grazie particolare a \texttt{Casa Terronia}, per chi c'� e chi c'� stato, per avermi accolto e coccolato, perch� non mi avete mai cacciato ma anzi per il dispiacere di quando non c'ero: grazie di cuore, siete state la mia seconda famiglia e per ognuna di voi ho messo da parte un \textit{grazie speciale}!

\paragraph{}
E poi c'� \textbf{lei}, ma qua grazie � poco. Ti ho conosciuta a un passo dalla tua tesi triennale, ricordi? Ora sei \texttt{dottoressa magistrale}, hai due tesi, ben due, mica una, eh! Per� oggi tocca a me ... Ora tocca a me. E poi c'� \textbf{lei}, tre anni fa ho iniziato quest'avventura chiamata universit�, � vero, ma ho iniziato anche un'altra avventura, nello stesso giorno ... Nessuno mi aveva detto che stavo facendo due dei pi� importanti passi avanti nella mia vita \dots Non lo sapevo, non lo immaginavo. Non era scritto in nessun manuale come fare con te, eppure ci sono riuscito, \texttt{per fortuna}!! A distanza di tre anni, la tesi triennale e poi ... E poi c'� \textbf{lei}, \textit{ancora al mio fianco}! Grazie di cuore, grazie perch� sei cos� e non posso trovare parole migliori per dirlo, grazie perch� non sei come ti sognavo ma perch� sei come non mi sarei mai neanche sognato di sognarti. Grazie per essere stata accanto a me e avermi sopportato, sostenuto giorno dopo giorno, ma soprattutto grazie perch� se guardo indietro vedo un passato stupendo vicino a te, per� se cerco di scorgere cosa mi aspetta per il futuro � bello scoprire che tu ci sarai, ancora, come ci sei stata fin'ora. Ti \dots vorrei dire grazie. Da \dots un posto speciale, dal profondo del mio cuore. \texttt{Grazie, di cuore, grazie.}
