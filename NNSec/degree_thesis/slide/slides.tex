\section{Introduzione}

\subsection{Panoramica}

%% SLIDE

\begin{frame}
 \frametitle{Motivazioni}
 \begin{block}{\textbf{SPEED} (\textbf{S}ignal \textbf{P}rocessing in the \textbf{E}ncrypt\textbf{E}d \textbf{D}omain)}
  Col progetto SPEED vengono avvicinati due mondi apparentemente distanti, come:
  \begin{itemize}
   \item<2-> \alert<2,2>{\textbf{Tecniche di \textit{signal processing}}}:
    \begin{itemize}
     \item Strumenti di classificazione dei dati in classi di appartenenza
     \item Percettrone, reti neurali feed-forward multi-livello
    \end{itemize}
   \item<3-> \alert<3,3>{\textbf{Tecniche di crittografia}}:
    \begin{itemize}
     \item Sistemi dalle interessanti quanto utili propriet� omomorfiche
     \item Cifrario di Paillier, generalizzazione di Damg\aa rd-Jurik
    \end{itemize}
  \end{itemize}
 \end{block}
 \begin{block}<4->{Il lavoro di tesi}
  Partendo dallo studio di un protocollo gi� esistente nella teoria:
  \begin{itemize}
   \item Algoritmi e procedure prendono vita sotto forma di classi e relazioni fra esse
   \item Viene realizzato un classificatore in grado di operare con dati cifrati
   \item Si ottengono reti neurali capaci di lavorare in dominio cifrato
  \end{itemize}
 \end{block}
\end{frame}


\section{Dalla Teoria \ldots}

\subsection{Il Problema}

%% SLIDE

\begin{frame}
 \frametitle{Scenario}
 \begin{columns}[c]
  \column{0.60\textwidth}
   \begin{block}{Gli attori\ldots}
    \begin{itemize}
     \item <2->Bob vuole mettere a disposizione una rete neurale opportunamente allenata
     \item <3->Alice vuole usufruire del servizio offerto da Bob per elaborare i propri dati
     \item <4->Bob e Alice non si fidano l'uno dell'altro
     \item <5->Non si pu� o non si vuole trovare ad una terza parte fidata per entrambi
    \end{itemize}
   \end{block}
  \column{0.35\textwidth}
   \begin{overprint}
    \onslide<2>\begin{center}\pgfuseimage{bob}\end{center}
    \onslide<3>\begin{center}\pgfuseimage{bobalice}\end{center}
    \onslide<4>\begin{center}\pgfuseimage{bobalicetp}\end{center}
    \onslide<5>\begin{center}\pgfuseimage{bobalicentp}\end{center}
    \onslide<6>\begin{center}\pgfuseimage{bobalice}\end{center}
   \end{overprint}
 \end{columns}
 \begin{block}<6->{\ldots E un caso concreto}
  Si immagini:
  \begin{itemize}
   \item Una rete neurale in grado di diagnosticare una malattia pi� o meno grave
   \item Un capo (o ex tale) di governo con sintomi particolari
  \end{itemize}
  La riservatezza dei dati acquista un valore inestimabile.
 \end{block}
\end{frame}

%% SLIDE

\begin{frame}
 \frametitle{Requisiti}
 \begin{columns}<1->
  \column{0.45\textwidth}
   \pgfuseimage{random}
  \column{0.50\textwidth}
   \begin{block}{Sicurezza per Alice}
    Risiede in quella del cifrario sottostante:
     \begin{itemize}
      \item Protezione dei dati forniti in ingresso
      \item Protezione dei risultati ottenuti
     \end{itemize}
   \end{block}
 \end{columns}
 \begin{columns}[t]
  \column{0.45\textwidth}
   \begin{block}<2->{Sicurezza per Bob}
    Consiste nel proteggere la struttura della rete neurale attraverso:
    \begin{itemize}
     \item<3-> Espansione tramite aggiunta di neuroni fittizi ai livelli intermedi
     \item<4-> Permutazione di neuroni in uno stesso livello intermedio
     \item<5-> Occultamento dello stato del singolo neurone intermedio
    \end{itemize}
   \end{block}
  \column{0.50\textwidth}
   \begin{overprint}
     \onslide<1>\pgfuseimage{blank}
     \onslide<2>\begin{center}\vskip4pt\pgfuseimage{nninit}\end{center}
     \onslide<3>\begin{center}\vskip4pt\pgfuseimage{nnfill}\end{center}
     \onslide<4>\begin{center}\vskip4pt\pgfuseimage{nnperm}\end{center}
     \onslide<5>\begin{center}\vskip4pt\pgfuseimage{nnsign}\end{center}
   \end{overprint}
 \end{columns}
\end{frame}

\subsection{Il Cifrario}

%% SLIDE

\begin{frame}
 \frametitle{Propriet� Omomorfiche}
 Il cifrario di Paillier ha caratteristiche molto utili e interessanti, in particolare:
 \begin{block}{Propriet� omomorfiche \ldots}
  Siano $ m_i $ messaggi in chiaro ($ i = 1,\ldots ,n $), $ c_i = E\left(m_i\right) $ la loro versione cifrata (di conseguenza, $ m_i = D\left(c_i\right) $), siano $ a_i $ un insieme di $ n $ valori interi, allora:
  $$ D\left(\prod_{i = 1}^n{c_i^{a_i}}\right) = D\left(\prod_{i = 1}^n{E\left(m_i\right)^{a_i}}\right) = \sum_{i = 1}^n{a_i\cdot m_i} $$
 \end{block}
 \begin{block}<2->{\ldots E reti neurali}
  Siano $ x $ un nodo nel $ j $-esimo livello e $ \bar{x} $ e $ \bar{w} $ i vettori di nodi connessi e pesi associati (di lunghezza $ n $), sia $ \bar{c} = E\left(\bar{x}\right) $. Il valore cifrato di attivazione $ d_x $ per $ x $ risulta da:
  $$ d_x = \prod_{i = 1}^n{c_i^{w_i}} \hskip8pt \Longrightarrow \hskip8pt a_x = D\left(d_x\right) = \sum_{i = 1}^n{w_i\cdot x_i}$$
 \end{block}
\end{frame}

\subsection{Protocollo}

%% SLIDE

\begin{frame}
 \frametitle{Il Protocollo}
 \begin{block}{Neuroni di ingresso}
  Per ogni neurone di ingresso $ i $:
  \begin{itemize}
   \item<2-> Alice cifra il valore in ingresso $ m_i $ con la propria chiave pubblica
   \item<3-> Alice invia il valore cifrato $ c_i = E\left(m_i\right) $ a Bob, il quale lo associa al corrispondente neurone di ingresso della rete neurale per l'elaborazione
  \end{itemize}
 \end{block}
 \vspace{\stretch{1}}
 \begin{overprint}
  \onslide<1>\begin{center}\pgfuseimage{step0}\end{center}
  \onslide<2>\begin{center}\pgfuseimage{step01}\end{center}
  \onslide<3>\begin{center}\pgfuseimage{step02}\end{center}
 \end{overprint}
\end{frame}

%% SLIDE

\begin{frame}
 \frametitle{Il Protocollo}
 \begin{block}{Neuroni intermedi}
  \uncover<2->{Per ogni neurone intermedio $ k $, Bob ricava il valore di attivazione $ z^\prime_k $ come segue:}
  \begin{itemize}
   \item<3-> Calcola il valore cifrato $ d_k = E\left(a_k\right) $ e genera in modo casuale $ t_j \in \lbrace -1, 1\rbrace $:\\ se $ t_j = -1 $ allora $ d^\prime_k = d_k^{-1} = E\left(-a_k\right)$, altrimenti $ d^\prime_k = d_k $
   \item<4-> Invia $ d^\prime_k $ ad Alice, la quale computa e ritorna: $ z_k = E\left(g\left(D\left(d_k\right)\right)\right) $\\
   \begin{itemize}
    \item $ g\left(a\right) $ funzione non lineare di attivazione del neurone (segno o sigmoide)
    \item Necessario ed unico punto di interazione fra le parti
   \end{itemize}
   \item<5-> Se $ t_j = -1 $ allora $ z^\prime_k = E\left(1\right)z_k^{-1} $, altrimenti $ z^\prime_k = z_k $\\
   (segue dalle propriet� di anti-simmetria della funzione $ g\left(a\right) $)
   %(infatti: $ E\left(g\left(a_k\right)\right) = E\left(1 - g\left(-a_k\right)\right) = E\left(1\right)E\left(g\left(-a_k\right)\right)^{-1} = E\left(1\right)z_k^{-1} $)
  \end{itemize}
 \end{block}
 \vspace{\stretch{1}}
 \begin{overprint}
  \onslide<2>\begin{center}\pgfuseimage{step0}\end{center}
  \onslide<3>\begin{center}\pgfuseimage{step1}\end{center}
  \onslide<4>\begin{center}\pgfuseimage{step2}\end{center}
  \onslide<5>\begin{center}\pgfuseimage{step3}\end{center}
 \end{overprint}
\end{frame}

%% SLIDE

\begin{frame}
 \frametitle{Il Protocollo}
 \begin{block}{Neuroni di uscita}
  \uncover<2->{Per ogni neurone di uscita $ j $:}
  \begin{itemize}
   \item<3-> Bob computa il valore: $ d_j = E\left(a_j\right) $, inviandolo poi ad Alice
   \item<4-> Alice ricava il valore di uscita del singolo nodo come: $ z_j = g\left(D\left(d_j\right)\right) $
  \end{itemize}
  \uncover<5->{\textbf{Nota}: I valori cos� ottenuti rappresentano il risultato della computazione tramite la rete neurale di Bob sui dati forniti in ingresso da Alice, avvenuta:
  \begin{itemize}
   \item Oscurando la rete neurale, preservandone la struttura interna
   \item Garantendo riservatezza per i dati di Alice
  \end{itemize}
  }
 \end{block}
 \vspace{\stretch{1}}
 \begin{overprint}
  \onslide<2>\begin{center}\pgfuseimage{step0}\end{center}
  \onslide<3>\begin{center}\pgfuseimage{step4}\end{center}
  \onslide<4>\begin{center}\pgfuseimage{step5}\end{center}
  \onslide<5->\begin{center}\pgfuseimage{step0}\end{center}
 \end{overprint}
\end{frame}


\section{\ldots Alla Pratica}

\subsection{Progettazione e Struttura}

%% SLIDE

\begin{frame}
 \frametitle{NNSec}
 Il software realizzato:
 \begin{itemize}
  \item Implementa praticamente il protocollo proposto
  \item Realizza un classificatore per dati cifrati
 \end{itemize}
 \vskip16pt
 Inoltre, propone caratteristiche aggiuntive fra le quali:
 \begin{columns}[c]
  \column{0.40\textwidth}
   \begin{block}{Domanda\ldots}
    \begin{itemize}
     \item \alert<2,2>{Parser integrato,\\ linguaggio specifico}
     \item \alert<3,3>{Comunicazione fra\\ le parti coinvolte}
     \item \alert<4,4>{Interazione multi-utente,\\ accesso concorrente}
    \end{itemize}
   \end{block}
  \column{0.50\textwidth}
  \begin{block}{\ldots E risposta}
   \begin{overprint}
    \onslide<2>
    \begin{itemize}
     \item Supporto attraverso:
      \begin{itemize}
       \item \textbf{JFlex}
       \item \textbf{JavaCUP}
      \end{itemize}
     \item Realizzazione di un analizzatore sintattico/lessicale
     \item Ideazione di un linguaggio ad-hoc per la descrizione di reti neurali
    \end{itemize}
    \vskip16pt
    \onslide<3>
    \begin{itemize}
     \item Implementazione di un modello distribuito client-server
     \item Uso della tecnologia \textbf{RMI} (Remote Method Invocation)
     \item Comunicazione basata su messaggi scambiati fra oggetti
    \end{itemize}
    \onslide<4>
    \begin{itemize}
     \item Strutture dati e algoritmi provenienti dalla teoria dei sistemi operativi
     \item Pattern di progettazione presi in prestito dalle tecniche di ingegneria del software
     \item Uso degli strumenti messi a disposizione dal linguaggio
    \end{itemize}
   \end{overprint}
  \end{block}
 \end{columns}
\end{frame}

\subsection{Reti Neurali in NNSec}

%% SLIDE

\begin{frame}
 \begin{columns}[c]
  \column{0.60\textwidth}
   In \texttt{NNSec} le reti neurali sono gestite attraverso generici modelli unici.
   \begin{block}{Pattern Composite}
    Espansione e permutazione delle reti neurali:
    \begin{itemize}
     \item Espansione durante la fase di composizione
     \item Permutazione prima di ogni richiesta d'uso
     \item Interfaccia di base unica per il client
    \end{itemize}
   \end{block}
  \column{0.35\textwidth}
   \begin{center}\pgfuseimage{composite}\end{center}
 \end{columns}
 \begin{block}<2->{Gestore delle Reti Neurali}
%  Il gestore delle reti neurali:
  \begin{itemize}
   \item Inserisce ogni rete neurale in un involucro che la espande
   \item Associa ad ogni rete neurale un semaforo che ne regola l'accesso concorrente
   \item Si preoccupa di forzare la permutazione dei neuroni
   \item Gestisce il recupero delle informazioni e le richieste d'uso
  \end{itemize}
 \end{block}
\end{frame}

\subsection{Modello di Comunicazione}

%% SLIDE

\begin{frame}
 \begin{overprint}
   \onslide<1>\begin{center}\pgfuseimage{blank}\end{center}
   \onslide<2>\begin{center}\pgfuseimage{qm1}\end{center}
   \onslide<3>\begin{center}\pgfuseimage{qm2}\end{center}
   \onslide<4>\begin{center}\pgfuseimage{qm3}\end{center}
   \onslide<5>\begin{center}\pgfuseimage{qm4}\end{center}
 \end{overprint}
 \begin{block}{I Quattro Moschettieri}
  Il cuore di \texttt{NNSec}, oltre che dal gestore delle reti neurali, comprende:
  \begin{itemize}
   \item<2-> \alert<2,2>{Factory remota}: Risponde alle necessit� di interazione
   \item<3-> \alert<3,3>{Lavoratori}: Servono richieste diverse in modo indipendente e concorrente
   \item<4-> \alert<4,4>{Modulo di comunicazione}: Impostazione d'ambiente, inoltro di richieste
   \item<5-> \alert<5,5>{Calcolatore}: Risolve il problema del riferimento circolare
  \end{itemize}
 \end{block}
\end{frame}

\subsection{Prove Sperimentali}

%% SLIDE

\begin{frame}
 \begin{columns}[t]
  \column{0.65\textwidth}
   \begin{overprint}
    \onslide<1->\pgfuseimage{allgraph}
    %\onslide<2>\pgfuseimage{bitot}
    %\onslide<3>\pgfuseimage{bicli}
    %\onslide<4>\pgfuseimage{biser}
    %\onslide<5>\pgfuseimage{allgraph}
   \end{overprint}
  \column{0.30\textwidth}
   \begin{block}{Test}
    Preparazione:
    \begin{itemize}
     \item Rete neurale (overfitting sui dati)
     \item Numero neuroni intermedi variabile
     \item Chiave di lunghezza 1024 bit
     \item Processore Quad Core (2.40GHz) e 4Gb RAM
    \end{itemize}
   \end{block}
 \end{columns}
 \begin{block}<2->{Risultati}
  I risultati ottenti determinano:
  \begin{itemize}
   \item<3-> Crescita lineare in base al numero di neuroni intermedi
   \item<4-> Esistenza di un punto di taglio con uguale distribuzione del carico di lavoro
   \item<5-> Degenerazione pi� consistente lato client
  \end{itemize}
 \end{block}
\end{frame}

%% SLIDE

\begin{frame}
 \begin{block}{Approfondimenti}
  Un aspetto in particolare merita di essere approfondito: la \textbf{degenerazione}
  \vskip8pt\uncover<2->{Motivazioni possibili della degenerazione:}\vskip4pt
  \begin{columns}<2->[t]
   \column{0.475\textwidth}
    \textbf{Lato Server}
    \begin{itemize}
     \item<3-> Costo dovuto ad operazioni di moltiplicazione e potenze
     \item<4-> Numero di operazioni superiore\ldots
     \item<5-> \ldots Ma di complessit� inferiore
    \end{itemize}
   \column{0.475\textwidth}
    \textbf{Lato Client}
    \begin{itemize}
     \item<3-> Costo legato principalmente alle operazioni di decifratura/cifratura
     \item<4-> Numero di operazioni inferiore\ldots
     \item<5-> \ldots Ma di complessit� superiore
    \end{itemize}
  \end{columns}
 \end{block}
 \begin{columns}
  \column{0.60\textwidth}
   \begin{block}<6->{Fattori}
    Alcuni dei fattori in gioco sono:
     \begin{itemize}
      \item Macchina virtuale (Java Virtual Machine)
      \item Costo in termini di operazioni macchina
      \item Architettura degli elaboratori
      \item Complessit� di cifratura/decifratura
      \item \ldots
     \end{itemize}
   \end{block}
  \column{0.35\textwidth}
   \begin{overprint}
    \onslide<6->\pgfuseimage{cmp}
   \end{overprint}
 \end{columns}
\end{frame}


\section{Conclusioni}

\subsection{Sommario}

%% SLIDE

{
\setbeamerfont{block title}{size=\Large}
\begin{frame}
 \begin{block}{Riassumendo}
  \begin{itemize}
   \item<2-> L'obiettivo � quello di:
    \begin{itemize}
     \item Avvicinare strumenti di classificazione e tecniche di crittografia
     \item Realizzare un classificatore per dati cifrati
    \end{itemize}
   \item<3-> La base di partenza:
    \begin{itemize}
     \item Interessanti propriet� omomorfiche del cifrario
     \item Un protocollo che sfrutti tali caratteristiche a suo favore
    \end{itemize}
   \item<4-> Il risultato finale:
    \begin{itemize}
     \item Un software sviluppato in Java che implementa il protocollo proposto
     \item Multi-utenza, accesso concorrente alle risorse
    \end{itemize}
   \item<5-> Le prove sperimentali hanno rivelato infine:
    \begin{itemize}
     \item Prestazioni accettabili in ogni caso su hardware non datato e in genere per reti neurali con un numero di neuroni intermedi contenuto
     \item Possibile previsione del comportamento in base al numero di neuroni
     \item Applicabilit� possibile (apparentemente) a scenari reali
    \end{itemize}
  \end{itemize}
 \end{block}
\end{frame}
}
