% <<<------------------------>>>

\section{Introduzione}

\subsection{Metodi di risoluzione simbolica}

%% SLIDE

\begin{frame}
 \frametitle{Panoramica}

 L'analisi simbolica come tecnica formale per lo studio di circuiti elettrici:
 \begin{itemize}
  \item Affonda le proprie radici in ambito accademico, dove nasce e si sviluppa
  \item Penetra nel mondo industriale, affiancandosi ad altri strumenti già presenti
 \end{itemize}

 \begin{columns}
  \column<2->{0.45\textwidth}
   \begin{block}{Vantaggi}
     \begin{itemize}
      \item Studio simbolico di un circuito
      \item Sintesi numerica a posteriori
     \end{itemize}
   \end{block}
  \column<3->{0.45\textwidth}
   \begin{block}{Svantaggi}
     \begin{itemize}
      \item Numero componenti utilizzabili
      \item Complessità computazionale
     \end{itemize}
   \end{block}
 \end{columns}

 \vspace{\stretch{1}}
 \begin{block}<4->{Idea}
  \begin{itemize}
   \item Spostarsi dall'ambito circuitale allo spazio dei grafi
   \item Sfruttare algoritmi specifici per \textit{risolvere} il circuito attraverso la sua immagine
   \item Ricavare una soluzione che sia \textit{valida} all'interno dello spazio di partenza
  \end{itemize}
 \end{block}

\end{frame}

%% SLIDE

\begin{frame}
 \frametitle{Il DET}
 L'Università di Firenze è molto attiva nell'ambito della simulazione simbolica:
 \begin{itemize}
  \item Con i suoi professori, tanto nella ricerca quanto nello sviluppo
  \item Con i suoi studenti, attraverso collaborazioni, progetti e lavori di tesi
 \end{itemize}

 \begin{columns}
  \column<2->{0.45\textwidth}
   \begin{block}{SapWin}
     \begin{itemize}
      \item Sviluppato internamente alla Facoltà di Ingegneria (DET)
      \item Ideato per piattaforme Microsoft Windows (portabile, con alcuni accorgimenti)
      \item Completo e funzionale
     \end{itemize}
   \end{block}
  \column<3->{0.45\textwidth}
   \begin{block}{QSapecNG}
     \begin{itemize}
      \item Nato come lavoro di tesi specialistica
      \item Supporto multipiattaforma (Windows, GNU/Linux, MacOS)
      \item Giovane e flessibile
     \end{itemize}
   \end{block}
 \end{columns}

\end{frame}

% <<<------------------------>>>

\section{Teoria}

\subsection{Sintesi circuitale}

%% SLIDE

\begin{frame}
 \frametitle{Componenti elementari}

 \begin{columns}[T]
  \column{0.3\textwidth}
   \begin{overprint}
    \onslide<1>
     \begin{table}
      \centering
      \begin{tabular}{l}
       \rowcolor{blue!20}
       \footnotesize Componente\\ \\
       \pgfuseimage{resistor}\\
       \pgfuseimage{capacitor}\\
       \pgfuseimage{inductor}
      \end{tabular}
     \end{table}
    \onslide<2>
     \begin{table}
      \centering
      \begin{tabular}{l}
       \rowcolor{blue!20}
       \footnotesize Componente\\ \\
       \pgfuseimage{nullor}
      \end{tabular}
     \end{table}
    \onslide<3>
     \begin{table}
      \centering
      \begin{tabular}{l}
       \rowcolor{blue!20}
       \footnotesize Componente\\ \\
       \pgfuseimage{opampl}
      \end{tabular}
     \end{table}
    \onslide<4>
     \begin{table}
      \centering
      \begin{tabular}{l}
       \rowcolor{blue!20}
       \footnotesize Componente\\ \\
       \pgfuseimage{vccs}
      \end{tabular}
     \end{table}
   \end{overprint}
  \column{0.1\textwidth}
   \begin{pgfpicture}{0cm}{0cm}{1.2cm}{1cm}
    \pgfsetstartarrow{\pgfarrowsingle}
    \pgfsetendarrow{\pgfarrowsingle}
    \pgfxycurve(0,0.5)(0.4,0.75)(0.8,0.75)(1.2,0.5)
   \end{pgfpicture}
  \column{0.5\textwidth}
   \begin{overprint}
    \onslide<1>
     \begin{table}
      \centering
      \begin{tabular}{cc}
       \rowcolor{blue!20}
       \footnotesize Grafo in corrente & \footnotesize Grafo in tensione\\ &\\
       \pgfuseimage{ijedge} & \pgfuseimage{jiedge}
      \end{tabular}
     \end{table}
    \onslide<2>
     \begin{table}
      \centering
      \begin{tabular}{cc}
       \rowcolor{blue!20}
       \footnotesize Grafo in corrente & \footnotesize Grafo in tensione\\ &\\
       \pgfuseimage{nullorgi} & \pgfuseimage{nullorgv}
      \end{tabular}
     \end{table}
    \onslide<3>
     \begin{table}
      \centering
      \begin{tabular}{cc}
       \rowcolor{blue!20}
       \footnotesize Grafo in corrente & \footnotesize Grafo in tensione\\ &\\
       \pgfuseimage{nullorgi} & \pgfuseimage{nullorgv}
      \end{tabular}
     \end{table}
    \onslide<4>
     \begin{table}
      \centering
      \begin{tabular}{cc}
       \rowcolor{blue!20}
       \footnotesize Grafo in corrente & \footnotesize Grafo in tensione\\ &\\
       \pgfuseimage{vccsgi} & \pgfuseimage{vccsgv}
      \end{tabular}
     \end{table}
   \end{overprint}
 \end{columns}
 \bigskip

 \begin{itemize}
  \item<1-> Una manciata di componenti rappresentano il nucleo del processo di sintesi
  \item<2-> La loro immagine nello spazio dei grafi è semplice e immediata
  \item<3-> Concorrono ad aggiungere archi e nodi in numero equivalente fra i due grafi
  \item<4-> Risulta necessario memorizzare gli archi nell'ordine di inserimento, associando loro le informazioni utili relative al singolo componente da cui derivano
 \end{itemize}

\end{frame}

%% SLIDE

\begin{frame}
 \frametitle{Componenti composti}

 \begin{itemize}
  \item<1-> Attraverso i componenti elementari se ne possono generare di più complessi
  \item<2-> I generatori controllati sfruttano ad esempio transconduttanze e nullori
  \item<3-> Ipotizzando l'esistenza di un generatore fittizio e regolando opportunamente i parametri di controllo, generatori di corrente e tensione non controllati possono venire a dipendere da esso
  \item<4-> Componendo ulteriormente si ottengono altri elementi derivati\\ (i.e. trasformatore ideale)
 \end{itemize}

 \bigskip
 \begin{columns}[T]
  \column{0.25\textwidth}
   \begin{overprint}
    \onslide<1>
     \begin{table}
      \centering
      \begin{tabular}{l}
       \rowcolor{blue!20}
       \footnotesize Componente\\ \\
       \pgfuseimage{cccs}
      \end{tabular}
     \end{table}
    \onslide<2>
     \begin{table}
      \centering
      \begin{tabular}{l}
       \rowcolor{blue!20}
       \footnotesize Componente\\ \\
       \pgfuseimage{ccvs}
      \end{tabular}
     \end{table}
    \onslide<3>
     \begin{table}
      \centering
      \begin{tabular}{l}
       \rowcolor{blue!20}
       \footnotesize Componente\\ \\
       \pgfuseimage{vcvs}
      \end{tabular}
     \end{table}
    \onslide<4>
     \begin{table}
      \centering
      \begin{tabular}{l}
       \rowcolor{blue!20}
       \footnotesize Componente\\ \\
       \pgfuseimage{transformer}
      \end{tabular}
     \end{table}
   \end{overprint}
  \column{0.1\textwidth}
   \begin{pgfpicture}{0cm}{0cm}{1.2cm}{1cm}
    \pgfsetstartarrow{\pgfarrowsingle}
    \pgfsetendarrow{\pgfarrowsingle}
    \pgfxycurve(0,0.5)(0.4,0.75)(0.8,0.75)(1.2,0.5)
   \end{pgfpicture}
  \column{0.55\textwidth}
   \begin{overprint}
    \onslide<1>
     \begin{table}
      \centering
      \begin{tabular}{l}
       \rowcolor{blue!20}
       \footnotesize Circuito equivalente\\ \\
       \pgfuseimage{cccseq}
      \end{tabular}
     \end{table}
    \onslide<2>
     \begin{table}
      \centering
      \begin{tabular}{l}
       \rowcolor{blue!20}
       \footnotesize Circuito equivalente\\ \\
       \pgfuseimage{ccvseq}
      \end{tabular}
     \end{table}
    \onslide<3>
     \begin{table}
      \centering
      \begin{tabular}{l}
       \rowcolor{blue!20}
       \footnotesize Circuito equivalente\\ \\
       \pgfuseimage{vcvseq}
      \end{tabular}
     \end{table}
    \onslide<4>
     \begin{table}
      \centering
      \begin{tabular}{l}
       \rowcolor{blue!20}
       \footnotesize Circuito equivalente\\ \\
       \pgfuseimage{transformereq}
      \end{tabular}
     \end{table}
   \end{overprint}
 \end{columns}

\end{frame}

% <<<------------------------>>>

\subsection{Metodo dei due grafi}

%% SLIDE

\begin{frame}
 \frametitle{Ricerca degli alberi di copertura comuni}

 \begin{block}{Definizione}
  Siano $gI^L$ e $gV^L$ due multiinsiemi di $M$ archi ciascuno per cui valga una corrispondenza biunivoca con $\mathbb{N}$, definiti su uno spazio di $N$ vertici. Vale:
  \begin{itemize}
   \item $gI^L$ e $gV^L$ definiscono due grafi $gI$ e $gV$ aventi entrambi $N$ nodi e $M$ archi
   \item Un albero di copertura comune è un insieme di $N-1$ indici $I^T$ tali da descrivere per i due insiemi un albero di copertura nel grafo corrispondente
  \end{itemize}
 \end{block}

 \begin{columns}[T]
  \column{0.45\textwidth}
   \begin{block}<2->{Algoritmo di Grimbleby}
    \begin{itemize}
     \item Ricerca \alert<2-2>{tutti gli alberi di copertura comuni} fra due grafi
     \item Sfrutta un \alert<2-2>{metodo inefficiente} di ricerca per \alert<2-2>{singolo albero}
     \item Costruisce nel caso peggiore \alert<2-2>{tutti i potenziali alberi}
    \end{itemize}
   \end{block}
  \column{0.45\textwidth}
   \begin{block}<3->{Metodo di Schach (aka MRT)}
    \begin{itemize}
     \item Migliora di un fattore che oscilla fra \alert<3-3>{$350\%$} e \alert<3-3>{$26.000\%$}
     \item Sfrutta un \alert<3-3>{metodo più efficiente} di ricerca per \alert<3-3>{singolo albero}
     \item Tratta i \alert<3-3>{ponti} e riduce il numero di potenziali alberi generati
    \end{itemize}
   \end{block}
 \end{columns}
 \bigskip

\end{frame}

% <<<------------------------>>>

\subsection{Rappresentazione simbolica}

%% SLIDE

\begin{frame}
 \frametitle{Circuiti $RCg_m$}

 Dato un circuito $RCg_m$, si ha l'insieme degli alberi di copertura comuni:
  $$ T = \{ T_1\ldots T_k \}~\mid~ T_i \in \{0\ldots M - 1\}^{N-1} $$
 Il determinante della matrice delle ammettenze ai nodi è calcolato come:
  $$
  \begin{array}{c}
  \Delta =
    \sum_{i = 1}^k{
      \varepsilon_i\ast\left(\begin{array}{c}
	prodotto~ammettenze~presenti~su~T_i
      \end{array}\right)
    }\\ \\
  \varepsilon_i = \left( det\left[ m\{ A_i \}\right]\right)\ast\left( det\left[ m\{ A_v\}\right]\right)~\in~\left[ 0,\pm 1\right]
  \end{array}
  $$
 Dove $m\{ A_i\}$ e $m\{ A_v\}$ rappresentano le matrici d'incidenza relative alla proiezione sui grafi in tensione e in corrente dell'albero di copertura comune in esame.

\end{frame}

%% SLIDE

\begin{frame}
 \frametitle{Estensione del modello}

 \begin{itemize}
  \item Circuiti contenenti componenti con rappresentazione impedenza
    \begin{itemize}
     \item Preso un generico albero $T$ su un grafo $G$, si definisce coalbero di $T$ la foresta $T'$ composta da tutti e soli gli archi non presenti in $T$
     \item Matrice $\Delta$ ottenuta come:
      $$
      \begin{array}{c}
      \Delta =
	\sum_{i=1}^n{
	  \varepsilon_i\ast\left(\begin{array}{c}
	    prodotto~ammettenze~presenti~su~T_i\\ \boldsymbol{e~impedenze~presenti~su~T'_i}
	  \end{array}\right)
	}\\ \\
      \varepsilon_i = \left( det\left[ m\{ A_i \}\right]\right)\ast\left( det\left[ m\{ A_v\}\right]\right)~\in~\left[ 0,\pm 1\right]
      \end{array}
      $$
    \end{itemize}
  \item Nullori e generatori controllati non di tipo transconduttanza
    \begin{itemize}
     \item Generatori controllati non di tipo transconduttanza vengono espressi tramite modelli equivalenti (circuiti $RCg_m$ e nullori)
     \item Non potendo collassare i nodi nei grafi in corrente e in tensione, il nullore impone di forzare in ogni albero di copertura comune archi posti fra i nodi stessi
    \end{itemize}
 \end{itemize}

\end{frame}

%% SLIDE

\begin{frame}
 \frametitle{Metodo del circuito modificato}

 Eccitando i nodi $i$/$0$ e ricavando parametri fra i nodi $j$/$0$, avremo:
  $$
   \begin{array}{c}
    Z_{in} = \frac{V_i}{I_i} = \frac{\Delta_{ii}}{\Delta}~\qquad~
    \frac{V_o}{I_{in}} = \frac{V_j}{I_i} = \frac{\Delta_{ij}}{\Delta}~\qquad~
    \frac{V_o}{V_{in}} = \frac{V_j}{V_i} = \frac{\Delta_{ij}}{\Delta_{ii}}
    \\ \\
    \begin{array}{c}
      \Delta = determinante~matrice~ammettenze~ai~nodi\\
      \Delta_{ij} = ij_{esimo}~cofattore~matrice~ammettenze~ai~nodi
    \end{array}
   \end{array}
  $$

 \begin{center}
  \begin{overprint}
   \onslide<1>{\pgfuseimage<1>{circbase}}
   \onslide<2>{\pgfuseimage<2>{circmod}}
  \end{overprint}
 \end{center}

 \uncover<2->{
  Per il circuito modificato si ottiene una separazione netta dei parametri, ovvero:
   $$ \Delta ' = \Delta + y'\Delta_{ii} + g'\Delta_{ij}$$
 }

\end{frame}

% <<<------------------------>>>

\subsection{Esempio}

%% SLIDE

\begin{frame}
 \frametitle{Esempio}

 Esempio d'analisi di un semplice circuito \alert<1>{divisore di tensione}.

 \begin{center}
  \begin{overprint}
   \onslide<1>{\pgfuseimage<1>{resol}}
   \onslide<2>{\pgfuseimage<2>{resoldip}}
   \onslide<3>{\pgfuseimage<3>{resoleq}}
   \onslide<4>{\pgfuseimage<4>{resolmod}}
   \onslide<5>{\pgfuseimage<5>{resolgs}}
   \onslide<6>{\pgfuseimage<6>{resol}}
  \end{overprint}
 \end{center}

 \begin{itemize}
  \item<2-> Tutti i generatori indipendenti vengono a dipendere da un generatore fittizio
  \item<3-> Eventuali elementi composti vengono rimpiazzati da modelli equivalenti
  \item<4-> Viene aggiunta la circuiteria d'ingresso per il circuito modificato
  \item<5-> Ha luogo la sintesi del circuito in forma di grafi in tensione e in corrente
  \item<6-> Si ricavano gli alberi di copertura comuni e quindi la formula di trasferimento
 \end{itemize}

\end{frame}

% <<<------------------------>>>

\section{Pratica}

\subsection{Note di programmazione}

%% SLIDE

\begin{frame}
 \frametitle{Cenni}

 \center
 Una breve panoramica in merito alle caratteristiche del software

 \begin{columns}[T]
  \column{0.45\textwidth}

   \begin{block}<2->{Framework}
    \begin{itemize}
     \item Boost C++ Libraries
      \begin{itemize}
       \item Boost Graph Library
       \item Property Tree
       \item Property Map
       \item \ldots
      \end{itemize}
     \item Qt Framework
    \end{itemize}
   \end{block}

  \column{0.45\textwidth}

   \begin{block}<3->{Tecniche di programmazione}
    \begin{itemize}
     \item Separazione backend/frontend
     \item Template metaprogramming
     \item Concept checking
     \item Tag dispatching
     \item Traits
    \end{itemize}
   \end{block}

 \end{columns}

 \begin{columns}[B]
  \column{0.45\textwidth}

   \begin{block}<4->{Design pattern}
    \begin{itemize}
     \item Template Method
     \item Monostate
     \item Observer
     \item Builder
     \item Facade
    \end{itemize}
   \end{block}

  \column{0.45\textwidth}

   \begin{block}<5->{Particolarità}
    \begin{itemize}
     \item \textbf{S}calabile
     \item \textbf{U}sabile
     \item \textbf{P}ortabile
     \item \textbf{E}volvibile
     \item \textbf{R}iusabile
    \end{itemize}
   \end{block}

 \end{columns}

\end{frame}

% <<<------------------------>>>

\subsection{Screenshot}

%% SLIDE

\begin{frame}
 \frametitle{QSapecNG}

 \begin{center}\begin{overprint}
  \onslide<1>{\pgfuseimage<1>{ss1}}\onslide<2>{\pgfuseimage<2>{ss3}}\onslide<3>{\pgfuseimage<3>{ss4}}
 \end{overprint}\end{center}

\end{frame}

% <<<------------------------>>>

\section{Conclusioni}

\subsection{Conclusioni}

%% SLIDE

\begin{frame}
 \frametitle{Conclusioni}

 SapecNG e QSapecNG sono soggetti principalmente a:
 \begin{itemize}
  \item Interesse interdisciplinare dal punto di vista teorico:
    \begin{itemize}
     \item Ambito elettrotecnico (i.e. analisi circuitale simbolica o semi-simbolica)
     \item Ambito informatico (i.e. teoria dei grafi, design di sistemi CAD)
    \end{itemize}
  \item Interesse trasversale fra campi apparentemente distanti:
    \begin{itemize}
     \item Mondo accademico (i.e. studio degli aspetti teorici, uso per scopi didattici)
     \item Modello industriale (i.e. applicabilità dell'analisi simbolica)
    \end{itemize}
 \end{itemize}

 \bigskip

 \begin{block}<2->{Il lavoro di tesi}
  Ha realizzato uno strumento effettivamente fruibile:
  \begin{itemize}
   \item Scaturito da studi e approfondimenti in merito agli aspetti teorici
   \item Con radici che affondano nell'ambito dell'elettrotecnica e dell'informatica
   \item In grado di dare una risposta alle necessità pratiche del modello industriale
   \item Capace di abbracciare richieste teoriche e didattiche del mondo accademico
  \end{itemize}
 \end{block}

\end{frame}
